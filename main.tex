\documentclass[12pt,lang=cn, mode=fancy, scheme=chinese, chinesefont=ctexfont]{elegantbook}

% ==================================================
% 封面与基础设置
% ==================================================
\usetikzlibrary{calc, math, decorations.markings}
\usepackage[normalem]{ulem}
\usepackage{tcolorbox}
\tcbuselibrary{skins, breakable}
\usepackage{fontawesome5}
\usepackage{etoolbox}
% \title{威海一中实验部生存指北}
% \subtitle{做题蛆自救指北} 
% \author{本书编写组} 
% \institute{威海一中实验部}
% \date{\today}
% \version{1.0}
% \linespread{1.5}
% % 自定义封面信息
% \bioinfo{关于本书}{为实验部,但不止于实验部}
% 
% % 如果没有封面图,请注释掉下面这行;如果有,请确保文件名正确
% % \cover{cover.jpg}
% \logo{logo.jpg}   

\begin{document}
\renewcommand{\today}{\number\year 年 \number\month 月 \number\day 日}
% 定义 Q 样式:只有左侧一条竖线
\newtcolorbox{questionbox}{
    blanker,            % 去掉所有默认边框和背景
    breakable,          % 允许跨页
    left=5mm,           % 文字距离左边缘的距离
    borderline west={1.5pt}{0pt}{structurecolor}, % 左侧竖线:宽度1.5pt,偏移0pt,颜色为主色调
    fontupper=\bfseries\color{structurecolor!80!black}, % 问题文字加粗,颜色略深
    before skip=1.5em,  % 与上方文字的间距
    after skip=0.5em    % 与下方回答的间距
}

% 定义 A 样式:简单的缩进或正常文字
\newcommand{\answer}{%
    \par\noindent\hspace*{5mm}\textbf{A:}\quad % 稍微缩进,开头有个 A:
}

% --- 请用这段代码完全替换原来的 myquote 定义 ---
\newtcolorbox{myquote}[1]{%
    blanker, breakable, left=1.8cm, right=1cm, top=1em, bottom=1em,%
    fontupper=\kaishu,%
    overlay={%
        \node[scale=6, text=structurecolor!20, anchor=north west, inner sep=0pt, xshift=5pt, yshift=-10pt] at (frame.north west) {``};%
    },%
    after upper={%
        \if\relax\detokenize{#1}\relax\else%
            \par\bigskip\hfill%
            \sffamily\small\color{structurecolor!80!black} --- \textbf{#1}%
        \fi%
    }%
}
%\maketitle
\begin{titlepage}
    % 1. 页面设置
    \newgeometry{top=4cm, bottom=3cm, left=3cm, right=3cm}
    
    % 2. 顶部标题区 (字体修改重点)
    \begin{center}
        % 主标题:New 序言
        % \heiti: 切换为黑体 (对应图片中的字体)
        % \bfseries: 加粗
        % \fontsize{42pt}{50pt}: 稍微加大字号,增强冲击力
        {\fontsize{42pt}{50pt}\bfseries\lishu\color{black!90} 实验部生存指北 \par}
        
        \vspace{0.8cm}
        
        % 副标题:生存手册
        % 同样使用黑体,去掉装饰线,显得更干净
        {\fontsize{24pt}{30pt}\bfseries\lishu\color{gray!80} —— 做题家自救指南 —— \par}
    \end{center}
    
    \vspace{1.5cm} 
    
    % 3. 核心图形区域 (洛伦兹吸引子)
    \begin{center}
        \begin{tikzpicture}
            % A. 插入 Python 生成的图片 (请确保 lorenz_attractor.png 在文件夹中)
            \node[anchor=center, inner sep=0] (image) at (0,0) {
                \includegraphics[width=12cm]{cover.png}
            };
            
            % C. 底部公式:洛伦兹方程组
            \node[below, opacity=0.7, font=\small, color=black!70] at (0, -4.5) {
                $\begin{cases} 
                \frac{dx}{dt} = \sigma(y-x) \\ 
                \frac{dy}{dt} = x(\rho-z)-y \\ 
                \frac{dz}{dt} = xy-\beta z 
                \end{cases}$
            };
        \end{tikzpicture}
    \end{center}
    
    \vfill
    
    % 4. 底部信息
    \begin{center}
        % 作者名也统一用黑体,风格更统一
        {\large\heiti \authorname 本书编写组 \par}
        \vspace{0.5cm}
        {\heiti\color{gray} \today \par}
        \vspace{0.2cm}
        {\tiny\color{gray!40} \scshape Version 1.0  \par}
    \end{center}
    
    \restoregeometry
\end{titlepage}
\frontmatter
\chapter*{声明}
\markboth{Introduction}{声明}
《实验部生存指北》(后称“指北”)是一本公益小册子。手册是两位作者根据兴趣自发撰写完成,版权属于本书编委会。在本书撰写过程中,我们始终以“中立”和“公益”为原则,没有接受其他任何组织任何形式的支持。未经编委会许可,任何组织或个人不得违反相应的版权条例抄袭、转载、摘编、修改本书内容;不得将本书用于商业目的;不得对本手册原意进行曲解、修改和未授权的大范围分发。

在编写本指北时,作者们听取了大量同学的意见和观点,并尽可地将各种观点统一在手册的框架下。这样做的目的,是希望能向读者传达更多可供参考的观点和意见。本手册并不构成任何明确的行动建议,因此作者不承担由此产生的衍生责任。

本指北作者不能保证手册内容中没有对其他组织的误解和偏见。指北内容的正确性并没有经过权威审查,指北作者无法保证指北中的方法始终有效。指北作者亦无力确认手册是否违反了读者所在地的各种法规,请各位读者参照当地行政规定。如有违反,请您停止阅读并立即销毁指北的任何副本。对于未经授权传播指北而造成的各种问题,指北作者概不负责。指北作者无法确定指北内容是否会对读者身心健康产生不良影响。如果您未满18岁,或因阅读指北而产生不适,请立即停止阅读并咨询心理医生。

本书编委会欢迎接受您的指教。如果您对本书内容有任何问题、或建议,请联络我们:survivewhyz@163.com。我们并不保证回复每一封邮件,但是我们会认真接受并思考您的意见,并在后续版本中做出相应的改进。

\begin{center}
    本手册为内部交流资料,仅供个人学习参考,严禁用于商业用途。
\end{center}
\begin{flushright}
野草 ~负熵~~~~~~~~~\\
\today
\end{flushright}


\chapter*{前言}
\markboth{Introduction}{前言}
作为小镇做题家群体中的一员,我和你们有着相同的困惑、迷茫,相同的痛苦和烦恼。在经历过这些苦痛和他人的善意后,我们决定抽出一部分时间来编写这本《实验部生存指北》来传递善意,帮助一下在高中的同学们,期望能够减少一些困惑和迷茫,提供一些切实可行的路径和方法。

虽然说叫做《实验部生存指北》,但并不是教授什么生存技巧和小聪明,而是以经历者的身份去和过去的自己对话,解答自己过去的迷茫,帮帮过去的那个可怜的自己。

同时,我们也对普通部同学表示关心。但因为没有体验过普通部的生活,所以无法提供有针对性的建议和帮助,只能希望普通部的同学们也能找到适合自己的方法,走出一条属于自己的路。也可以参考其中大部分内容。

这本书的封面经过我们用心的设计。这个图形是著名的洛伦兹吸引子,形状像是一只巨大的蝴蝶,也是著名的“蝴蝶效应”的起源:用来描述混沌系统(非线性系统)的初值对于其后续变化的巨大影响。

我们在这里使用这个图形,是因为我们希望每个同学都能在高考这个混沌的非线性系统中,能像蝴蝶一样振翅飞翔。

我们不指望这本手册对每个同学都适用,只是希望能或多或少帮上一点忙。

因为本书是为高中生而写,于是插入了一些小彩蛋,可能存在遣词造句不太规范的地方,请见谅。同时,我们体贴的为大家在章节刚开始的地方插入了名言,希望能为大家带来一些启发和力量。

祝你好运。

另外,本书的\LaTeX 模板为\href{https://github.com/ElegantLaTeX/ElegantBook}{ElegantBook},感谢 \href{https://github.com/EthanDeng}{EthanDeng}提供的模板。
\begin{flushright}
野草 ~负熵~~~~~~~~~\\
\today
\end{flushright}

\chapter{序言}

你好。

当你翻开这本手册时,你可能正身处教室的某个角落,周围是堆叠如山的试卷和低头赶路的同侪。我们不仅知道你在经历什么,更理解这种经历背后的结构性困境。

这是一本“生存手册”,而非“成功学指南”。我们不贩卖“只要努力就能上清北”的廉价幻觉,因为事实是:在教育资源不均衡的今天,大部分人只有拼尽全力才能上个好大学。高考作为一种筛选机制,其本质是残酷的零和博弈。

但我们编写这本手册,是因为我们坚信:在数据化的排名和标准化的考卷之外,你仍是一个拥有体温、情感和无限可能性的具体的人。

这里有简练的学习方法用来抵御现实的碾压,因为毕竟高考是唯结果论;但这里更有关于“如何保持心理韧性”、“如何正确看待爱与被爱”、“如何不因匮乏感而自我贬低”的温暖防线,因为这是你心里的锚,防止你在分数的洪流中迷失自我。

请记住,高考是你人生中重要的一战,但它不是你人生的全部战役,更不是你作为“人”的最终判决书。


回首三年生活,尽管经历过的失败,迷茫和崩溃数不胜数,但同样快乐,自由和难以忘怀的记忆还是占据了更多的份额。我不希望大家在高中走我们同样走过的弯路,在同样的迷茫面前止步,在长夜中难以入眠;也不希望大家成了制度的牺牲品,成了失去自我的机器,成了毫无头脑的人。

愿你在这三年里,不仅学会如何做题、打怪升级,更学会如何作为一个完整的人,去呼吸,去思考,去爱。

\tableofcontents

\mainmatter
\chapter{定位与博弈 (The Context)}

\section{全景图}
各位同学你们好,首先恭喜你们通过了中考,来到了拥有威海最好教育资源之一的威海一中实验部。

但抱歉,威海一中在事实上和其他任何一个山河四省的高中没有任何区别,是一个培养做题家的机器。幸运的是,这里有一定的人文关怀。你们接下来的三年不用在衡水模式中挣扎,能过度过一个相对没那么高强度的三年,有一些值得思念的晚霞和回忆;不幸的是,你们要和同省内其他卷生卷死的同学一起pk高考,而威海的教育资源相当匮乏——这意味着即使考上了实验部,也不代表你一定能上一所好(985、211、其他双一流)学校。

数据显示,2025年山东高考人数101万人,而作为对比,吉林省只有12万人。这意味着,生在了山东,你就需要付出更多的代价才能考上同样的一所学校。根据预估,高考人数将在2034年达到顶峰——1697万人。我只能说,祝你好运。

但同样的,我们必须清楚,高考能给你带来什么,这是必要的吗?对绝大部分普通人来说,是的。高考能给你一个平台以几乎公平的方式,与全省的人进行排名,来决定你最后的去处——大学。大学在很大意义上是你“后半生的起点”,当然,不是终点。

这意味着大学的高度很大程度上决定了你在大学期间和大学毕业后的平台高度。举几个简单例子:选择了非清北、华五陆本就已经距离美国 phd十万八千里了;而普通一本和211、985之间的差距就更大了——有些(985)的保研中,初筛就不会通过非985的申请书;而尽管你和你的末9大学在网友的口中是依托答辩,但在真正的工作面试中,985这个头衔就已经说明了太多、战胜了太多非985的申请者。

如果是做研究,同样地,由于现在科研圈的内卷,大家其实更关注第一学历:即本科学历。第一学历歧视的例子比比皆是,譬如马琰明院士去浙江大学担任校长的时候,第一学历(延边大学)还被拉出来鞭尸了一顿。所以在高校科研圈中,第一学历同样是很重要的。事实上,第一学历被某些人用来判定智商和行为能力。

当然,实现梦想和高学历不是充要条件,顶多学历高会更容易一点实现梦想。我们不说鸡汤,而是现实主义地角度出发,告诉大家高学历是更容易实现梦想的一条路:衣衫褴褛也有能拜佛西藏者,装备齐全也有半路而回的人,但现实就是学历更高更容易实现梦想,就这么简单。

尽管我们写这篇指北的初衷是为了帮助大家发掘做题以外的世界,但很遗憾,高考的选拔规则和如今就业形势的经济基础决定了所谓的“唯分数论的做题家思维”的上层建筑。所以,在高中和高考规则下,最后决定你去处的还是\textbf{分数}。这意味着你在高中的绝大部分时间都是在学习,这当然是不能避免的。

很抱歉,这就是现实。

我们在在这个章节无意去探讨高考是否公平,考这些科目的意义或其他,我们能做的就是在现有规则下尽量的做到最好。

\section{身份认同}

\begin{itemize}
    \item \textbf{解构“小镇做题家”}:\\
    这些年来,大家的自嘲越发极端,从“奋斗者”到“小镇做题家”再到“做题蛆”,代表了大家的自嘲和无奈,也是对于传统的努力叙事的解构。我们对它再进行解构:这算是一种资源匮乏下的奋斗形态,没有必要羞耻。
    
    \item \textbf{除了排名,你是谁?}\\
    在高中,绝对不要把所有的时间都投入学习——这会让你的精神高度紧张和压抑。不管多忙,都要留出一些空闲的时间来做那些看起来无意义的事情:读书、打球、魔方、弹琴,什么都可以,除了学习。
\end{itemize}

高考要筛选的绝对不是只会读死书的人,要保持自己的节奏和心理素质。如果把所有的时间都投入学习,不仅效率会降低,还大概率会导致考前紧张、容易崩溃。

想象你的自我是一个支撑生活的桌子。如果这张桌子只有一条腿“成绩”,那么一次月考的失利就能让你的整个世界崩塌。但如果你有四条腿——“成绩”是主腿,“篮球”“钢琴”“写日记”是另外三条——当“成绩”这条腿暂时断裂时,其他三条腿依然能支撑你站立,不至于让你在精神上粉碎性骨折。

这部分我们会在后面第五章详细谈到。

总之,在高中三年,不要把你作为“人”的\textbf{全部}价值,都通过“排名”这一种单一货币来衡量。

\section{异化的同学关系}
在这一节的开始之前,先允许我向大家介绍一下我们高考的选拔机制。首先先要明确的第一点是高考是一个全省竞争排位的模式,而非全国。诚然有的地区有的省份竞争较弱,试卷较为简单,但他们与我们并不构成直接竞争关系,与你构成竞争关系的只有山东本届的考生。

这一点信息我认为会给我们带来这几点启发:第一,尽可能去屏蔽那些吐槽京,沪等地高考简单而山东地狱难度的情绪,因为我们现在必须承认的一点就是我们无法改变这一切,我们就是生在了这样一个高考人数多,难度大的省份,然后呢?从此一蹶不振躺平摆烂,大骂资源不平等?况且你的对手也不是这帮人,而是与你考同一份试卷的“老乡”,这样的情绪对你的备考没有任何作用,只会徒增你的烦恼,降低你的斗志。第二,若你是高三考生,在做外省优质的模拟试卷时,不要被它地的分数线所吓怕,因为它没有任何参考意义,就拿浙江举例,前10\%的人可以赋分到A档(大于等于91分),而在山东只有3\%的人才能到A档。

所以不要太在意外省的分数线,更多的去关注一下本省的模拟划线。

在省内如何判断一所大学是否录取你呢,想必大家也都很清晰了,通俗的来说就是排位,通过你高考位次决定着这所大学这个专业是否有足够的名额来能包含住你。它与分数线无关,不是你考了多少分就会上什么大学,哪一年题难了分数线就低了。除非一些热门学校的热门风口专业可能会出现录取位次超级靠前。但其他专业基本会稳定在一个区间上,拿山东大学数学类举例,虽然21—24年分数线变化很大,位次基本就是在4000-4500之间,变化幅度很小。所以这告诉我们我们大可不必因为一次考试分数没达到预期而焦虑自责,更多去关系一下自己的排位,对于心仪的院校专业,可以通过往年的位次来划定目标,再换算到全市,再到全校,这来的更加实在。

好了,相信我简略的将高考的选拔机制给说明白了,前面介绍这么多,不知道大家有没有发现盲点,这和标题有什么关系?大家都是从初中一路领先,或靠努力,或靠自己的才华来到了威海高中教育的顶端了,想必大家也都听过什么“提升一分,干掉千人”或者“卷死身边同学”。大家心中肯定不服为什么我的同桌进了年级前20,我只能在年级30名,然后疯狂内卷竞争甚至对身边同学处处提防,别人做错题了也要对他冷嘲热讽,不会做的问题也不好意思虚心求教。但通过我的介绍不知道大家发现没有,高考的竞争确实是你排位高你就可以去好大学,但不是这么笼统。

这里我想引入一个概率视角的思考。高考虽然残酷,但它是一个大数法则起作用的“宏观战场”。你的竞争对手是谁?是全省几十万考生中,那些和你分数咬得最紧、甚至同分竞争排位的陌生人。请大家理性计算一下:你和你的同桌,恰好报了同一所大学、同一个专业,且分数恰好卡在边缘线上,导致他进了你没进的概率有多大?这个概率在统计学上几乎可以忽略不计。把全省范围的“宏观竞争”,错误地投射到班级内部的“微观竞争”,这是我们最大的认知误区。你的同桌考得好,挤掉的是外市、外校的某个陌生人,而不是你。为了这几乎为零的“撞车”概率去防备朝夕相处的战友,这在策略上是极低效的。

举个例子你要学数学,你同桌要学物理,那你俩完全不存在竞争关系啊,你投档的专业不包含物理,那你俩就不会“撞车”。谈何竞争。就算你俩专业重叠程度高,这同样也不意味着你们要竞争在一块,因为你们两个所向往的城市或者大学可能不一样,不填报一所大学,你俩竞争在哪。就算你俩投档投的是一个志愿,在我看来也是不要紧的,对于非上游985的学校,学校放出的名额不会少了,你俩相互鼓励,共同进步,上一所大学,大学多个交心朋友是什么坏事。就算你俩成绩都很优异,都能去“华五”,名额虽然少,你俩仍可以互相帮助,最终一个能去到“复旦”,一个去到“交大”,我觉着这也是顶峰相见吧。

综上,论述了这么多,我想跟你说的意思就是,不要异化你与你高中同窗(甚至是威海其他高中的同学)的关系,不要把他想象成一个竞争者,而是同行者,是伙伴是战友。而非处处提防。学校排名设置的意义仅仅是让你了解一下自己的水平,看看自己对知识的掌握程度如何,而非我比你排名靠前,你就被我踩在脚底下的胜利,你的竞争者不在你们班上,不在实验部,更不在威海一中。所以对于你的同学不仅不应提防,我更建议大家主动去给同学讲题。这绝不是在做慈善,而是为了你自己。学习金字塔理论告诉我们,“听懂”只能留存20\%的知识,而“教会别人”能留存90\%。当你给同学讲题时,你需要调动整个知识网络,把模糊的逻辑理顺,把卡壳的地方打通,这个“输出倒逼输入”的过程,就是最高效的复习。此外,还有一个“环境场效应”。如果你的同桌是大佬,你一定要希望他更强,因为一个高水平的班级环境会通过日常讨论、难题切磋,把你也不知不觉带到一个更高的水位。水涨才能船高,在一个平庸的群体里做第一,远不如在一个顶尖的群体里做凤尾更有价值。

\section{与老师的关系}
本节声明:不同老师有不同性格,请勿套用;本节建立在你对老师品格和素养的信任这一前提下。

在从小到大的小初高乃至大学,与老师的关系是绕不过去的话题。有的人和老师形同陌路,有的人和老师关系密切,这一切代表着什么,有必要“经营”这一段关系吗?

首先打破一个误区:不要把老师神圣化,也不要妖魔化。老师不同于传统意义上的长辈,也不是“权威”。通常相处(高中)可以以亦师亦友的方式和老师相处。简单来说,形式上不用过于拘束:没必要非常紧张、没必要想东想西;可以以一个严肃而活泼的方式进行:即身份上你要认清,老师是老师,是传授知识、在学校管理你的人;但同时可以以轻松而不轻浮的语调和老师交流、开玩笑,来拉进师生关系。

\textbf{1. 有必要吗?}

对高中来说,是比较有必要的。在最高压的三年里,你也不想一进教室就看见一张和你关系紧张的人的脸吧?和老师处好关系,其实也是一种对教师职业的祛魅,能够锻炼你的社交能力;其实也是多交了一个朋友,回到家还有个能去玩的地方;又或者在压力大的时候有一个能哭诉的去处。(笔者高三就经常去找老师聊天。)

当然,从利益关系的角度其实是没必要的,你把ta当NPC或许才能最大化“收益”,但现实不是游戏,人也不是机器,和老师的关系其实和同学处好关系没什么区别:或许你会收获一些利益,但这并不是我们做这件事情的理由。

一个常见的行为是,很多成绩优异的同学觉得自己的成绩全都是靠自己得来,和老师关系不大;又或者觉得老师没有出力。那我认为起码要维持“过得去”的程度,不至于让两方都难堪。说到底,这也只是社会关系的一种,如果你实在觉得处不来,没有必要硬处。

还有一个误区是所谓“人情世故”和“情商”云云都是糟粕,不应该关注。但事实上,不论是在学校、科研、公司,只要是有人的地方,情商就是很重要的一环。怎么与人交流交往,怎么保持分寸,怎么保持自尊,怎么拒绝他人——这些是只要你需要和他人交流就会用得到的东西,提前锻炼一下也不是坏事。

\textbf{2. 怎么“刷好感度”?}

最简单的方式,是用好的表现来做:要么是性格上,严肃活泼;学习上,认真专注;为人处事上,周到而不圆滑;交流沟通之后,能反思能改善。看起来,这些要求有点高,但本质上还是:“态度”。你可以学习不是最顶尖的,但力求是最认真的;逢年过节不用发最华丽的话,但力求有心;交流时不用非得托底交心,但指出问题你要改。老师也只是一个勤勤恳恳工作的普通人,看到了认真努力的苗子也会想帮一把;有人认可他的工作他也会很开心。如果你学习优秀,但太严肃认真,老师大概率会赏识你,但你们的关系不一定会很融洽。

当然,这些事情是“要做”,相反的,还有一些“不要做”:首先是不要“放弃自己”:老师说你也不听,给你指出问题你看到了也无动于衷,很难有人会坚持下来,很容易被老师放弃;第二,不要以任何形式 bully 同学:起外号、打架等;第三,犯错了不要逃避,敢作敢当。

\textbf{3. 简单实操}

如果你尽力做到了上面第二点的要求,那老师对你的印象分肯定不低。如果你再经常下课去问问题、平常和老师聊聊天,那很容易就关系很好了:只是老师碍于身份,肯定不能和你表现得太好;又或者觉得其他同学会觉得不公平等等,总之很难在学校里体现的特别明显,但肯定是能察觉的程度。

\section{与家长的关系}

我不得不承认,这一部分是难度最高、不可控因素最大的一章节,也是适用性最差的一部分,我只能姑妄言之,希望有所帮助。

家长常见的行为和心态是,要么漠不关心,要么就更加焦虑。前者反而还好,后者往往只能帮倒忙。家长总说孩子不能理解家长,但家长往往也不能理解孩子,这才是常态。如果你想改变你的父母,让原本不理解的他们理解你、支持你,那我劝你最好死了这条心:改变自己尚且不易,哪里有能力去改变他人呢?

我的建议是,少提起哪些容易起争端的话题;如果你的父母总是念叨,那最好家里都少待:能去自习室就去自习室,能呆在屋子里就呆在屋子里。少起冲突、少发脾气的秘诀只能是少沟通。

如果你的父母是可以沟通的,那就找他们聊一下你的情况和境遇,要求他们少输出情绪,甚至少输出道理,这样才能最大程度的减少家庭环境对你学习的影响。如果没法沟通,那我也没有办法。


% ==================================================
% 3. 技术与战术 (The Tactics)
% ==================================================
\chapter{技术与战术 (The Tactics)}

\section{学习概述}

\subsection{高考简述}

5年山东高考数据显示,前5\%大概590分上下,前1\%大概635+。实验部的前50的平均水平差不多可以达到1\%,即全省6000-7000名左右,即可以上一个末9的好专业,中9的一般专业或更好一点学校的差专业,努力一下可以达到华五水平。

山东省2025年高考总人数101万人,其中夏季高考的74万人(部分提前录取)。竞争是残酷的,但同样的,也会有一些其他的“福利”。

普遍来说,在科研导师招收学生的时候,会考虑“刻板印象”:山东的都勤奋刻苦,皮实耐用;在竞争的时候,由于已经习惯了高强度的竞争,有时也会很具有竞争力。当然,这些只能算是附带品。

\subsection{学习方法概述}
学习方法在具体实操上因人而异,但总归有几点共性。第一是效率,第二是独特性,第三是休息。

平常在学校里面,要求跑操前的2分钟像模像样地背诵,(对我来说)这毫无疑问是自我感动,将学习和痛苦挂钩;或者每次都熬夜到1点,在白天上课犯困睡觉的;或者在各种视频平台上挂着“xxx 博士学习 13/15/17小时的一天”的vlog,让人哭笑不得看了就难受:你刚开始或许会感叹一下他们的毅力,但随之而来的就是质疑:这些小时里面,真正有用的才多少呢?这种堆时间的工作方式是最低效的、最自我感动的,也是最“悲壮”的学习方法。我不想嘲讽他们,只是这样的方式对大部分人是无意义的。

\textbf{1. 我们必须承认,注意力是一种稀缺的、有限的资源。}

我们要保证效率,就是因为我们的目的是掌握知识、学会考点、知道怎么做题/知识点,而如果只是形式上重复而没有理解,就是无用功。

\textbf{2. 我们必须承认,每个人的学习方法都是不同的。}

我在高中见过记录的笔记精美,字迹优雅的同学考入华五,也见过几乎不记笔记,或形式很丑的同学名列前茅。说到底,学习方法不过是帮助你掌握知识的方法,没有必要追求那些高大上的,符合自己的、最有效率的才是最好的。所以要反对一切形式化的学习方法\sout{(比如跑操前假模假样拿着书实际上在唠嗑的行为)}

\textbf{3. 我们必须承认,身体是有极限的。}

身体有极限,身体状态会切实地影响学习状态。我们必须保证充足的休息和睡眠,以及适当的运动。这对你的身体健康和学习状态都很有帮助。切记不要忽视这一点!

\subsection{考试的艺术}
读到这里,我希望你重新审视一下自己的心态。请问你是否有以下几种心态:
\begin{enumerate}
    \item 考试在中档题上碰壁,不甘心决定死磕到底
    \item 对于压轴题有很强的执念,做不出来就会很难受,考前也会紧张于自己是否能解决它
    \item 对于压轴题有畏惧,如果时间紧迫甚至整道题都会放弃
\end{enumerate}

若你确实有这些问题,那么恭喜你,你将在这一节中看到解决方案,以及跟随我理性的分析,转变你的想法。

首先,我们要做的第一件事就是摆正高考的定位,那就是高考不是什么智商测试,也不是单科选状元,更不是证明你无比聪明能做出压轴大题。我们经历的考试其实是一个“资源配置”的过程,即考试,从来不是让你把所有题都做对,而是在有限的时间内,把你的知识变现率最大化。说的高大上一点就是我希望大家抛弃“做题”这个低维度的概念,我们将高考看作一个标准的数学模型——约束条件下的最优化问题。(说的简单一点就类似于给你一个指定函数我们来找到它的极值)

那么接下来,我也将引入目标函数:$Max(S)$,即最大化卷面总分。但这个函数受到两个刚性约束条件的限制:时间约束$T$:120分钟(或90分钟)是不可逾越的物理上限。脑力约束$E$:你的专注力和运算能力随时间呈非线性衰减。大多数考场悲剧的发生,不是因为能力不足,而是因为求解这个最优化问题时,陷入了“局部最优陷阱”,从而丢失了“全局最优解”。通俗的来说就是大家可能死磕某一道题,而全然不管后面的问题如何如何。

那么我将根据严谨的数学来——说服你上面出现的三种问题。

\textbf{第一个问题:警惕沉默成本}

很多同学在解题受阻时会产生赌徒心理:“我都算了5分钟了,现在放弃这5分钟就白费了。”这是典型的非理性偏差。沉没成本与未来的决策无关。无论你之前投入了多少时间,只要下一分钟产出分数的期望值低于平均水平,就必须立刻停止。

我们来构造一个数学模型:
\begin{definition}{沉没成本模型}{}
设 $S(t)$ 为你在某道题上投入时间后能获得的预期分数。
我们需要关注的不是 $S(t)$ 本身,而是它的导数 $S'(t)$,即边际得分率(分/分钟)。它代表了你当下每多投入1分钟,能带来多少分数的增量。
\end{definition}

假设全卷总分150分,时间120分钟。
你的理论平均得分率(基准线)是:$150/120 = 1.25$ 分/分钟。
这意味着,你在考场上的每一分钟,理论上都应该至少产出1.25分,否则你就是在亏损。

让我们看看一道难题的 $S'(t)$ 是如何变化的:
\begin{itemize}
    \item 初始阶段(0-3分钟):你正在通过阅读题目、建立坐标系、列基础公式来推进解题。此时你的思维是活跃的,获得部分步骤分的概率很高。此时,$S'(t) > 1.25$。(做这道题是划算的)
    \item 停滞阶段(3-X分钟):你卡住了。你的笔停了,脑子在空转,或者在草稿纸上重复无效计算。此时,你获得新分数的概率急剧下降。此时,$S'(t)$ 迅速跌落并趋近于0。
    \item 决策点(交叉点):当 $S'(t)$ 曲线向下穿过 1.25 分/分钟的那一刻(通常发生在卡壳后的第1分钟),就是数学上的绝对止损点。
\end{itemize}

假设你某道小题已经花费了4分钟,但对于这道题仍然没有什么头绪时,你应该及时止损。你可能会觉着我好像亏了,这是正常人的心理;但你要明白,你当下的决策不应该基于过去,应该基于的是为未来的考虑。那过去的4分钟,无论你继不继续做,它都已经消失了。它对于你下一分钟能得多少分,没有任何贡献。所以,在考场上,每隔一段时间就对自己执行一次“归零”:假设我是刚拿到卷子,看到这道题卡住了,我会怎么做?答案显然是:做下一道题。

\textbf{第二个问题:不要太把压轴题当回事}

首先,你必须敢于抛弃“理科偶像包袱”,你现在需要做的就是,大大方方的承认我就是有可能做不出来最后一问,数学,物理我就是可能考不了满分,145甚至135以上。但那又如何能。

我曾经也是如此,可能因为感觉做出大家做不出来的题是一件开心的事,也可能觉着做出来了就是体现我对数学的热爱。我花了大量时间去练习,虽有收获,我绝大多数时间压轴题都是迎刃而解,但这是非常低效的,我把大量时间用在了这上面,文科一团糟。成绩始终无法提升。而且你有了这种心态,你就会必然在数学等理科考试前无比紧张,畏手畏脚,似乎始终被什么东西困住一样,无法展示自己真正的水平,在无法解决时心情也会糟糕更会影响其他科目的考试。

而当你在大学中学习真正的数学,分析学epsilon-delta语言的严谨形象,代数学结构的美秒,数论等证明的巧妙与高深,这无不让你感叹过去高中时的一叶障目,一味的追求那高中压轴题的“奇技淫巧”实属浪费时间。

那么在考试中我们该如何正确看待压轴题目呢,我希望你始终记住,考试时尽可能做到利益最大化(分数最大化)。

我们来建立一个数学模型来推导一下:

\begin{definition}{压轴题决策模型}{}
\textbf{模型推导:} 设题目的分值为 $V$,预计耗时 $t$,成功率为 $P$。那么这道题的得分效率为:
\[ E_i = P_i \cdot V_i / t_i \quad (i \text{为试卷第} i \text{道题}) \]
\textbf{决策逻辑:} 考场上的每一秒,你都应该在该时刻选择 $E$ 值最高的题目去做。
\end{definition}

\textbf{实战应用:} 压轴题最后一问通常是 $V=4, t=15, P=20\%$,计算得出 $E=0.05$ 分/分钟;而检查前面的基础题,可能是 $V=5, t=2, P=90\%$,计算得出 $E=2.25$ 分/分钟。
\textbf{结论:} 在时间不足的情况下,去做压轴题最后一问在数学上是绝对亏损的策略。请永远优先选择那些 $E$ 值更高的区域。

\textbf{第三个问题:在压轴题上“薅羊毛”}

大家对压轴题(导数、圆锥曲线)往往有一种恐惧,要么不做,要不就想一口气做完。但从得分角度看,压轴题的每一问都是独立的计分点。

\begin{itemize}
    \item 第一问:通常是送分题(求导、求离心率等),难度甚至低于前面的填空题。这几分必须拿,无论后面的题有多难。
    \item 第二问:往往是分水岭。能做多少步骤就写多少步骤。高考阅卷是踩点给分,写出关键公式、列出方程组,哪怕最后算不出结果,你也能拿到一半的分数。
\end{itemize}

所以,对于压轴题而言,请把心态从“解题”转变为“抢分”。只要你把题目中的条件翻译成了数学公式(设直线、求导、列方程,韦达定理),你就已经完成了任务。不要因为算不出最终结果而羞愧,在只有1\%的人能算对的题目里,拿走50\%的过程分,难道你还不满足?

综上所述:
高考第一步就是知识上的大比拼,也就是基础题谁做的又快又好。第二步,就是策略问题了,这就是我前文所说的资源配置问题,谁能“博弈”出最优解,谁就能充分发挥自己的实力。当你认清了高考这一点本质的时候,你就已经赢在起跑线上了!

\section{拒绝“伪勤奋”也要拒绝”状态”借口}
上一节我们已经对部分人的“勤奋”祛魅了,很多同学看到了伪勤奋会产生一种矫枉过正的心态:既然我累了,那么累了状态不好,就没什么效率,那我这和伪勤奋有什么区别,那我还不如不学了,我去睡觉,去打会游戏,等精力回满100\%我再学。但是,在高强度的备考下,这可能是一个陷阱!

上述的这种心态看似逻辑链条完整,论证具有说服力,但这里其实是有漏洞的,也就是我们仅仅定性的去分析这一个心态,但并没有定量的去分析。就像你不能仅仅说一件事有可能发生又有可能不发生那么发生的概率就是50\%。

我来给大家算一笔账。我们现在有两个策略
\begin{itemize}
    \item \textbf{A完美主义者}:我一天只在最好的状态下学习。假设全神贯注的时间为2小时,此时效率为100\%,那么我们可以计算得出我们的产出为 $2 \times 100\% = 2$。
    \item \textbf{B实用主义者}:我就算状态一般也要尽力去学一会。假设这样的时间为6h,此时我们退一万步说效率只有50\%,那么我们通过计算可以得到我们的产出为 $6 \times 50\% = 3$。
\end{itemize}

结论很是残酷:B策略的结果就是要好于A,在高考这种需要海量记忆和刷题的考试中,“量”本身就是一种“质”。只有高强度的思维爆发是不够的,你还需要漫长的、枯燥的、甚至稍显低效的重复来夯实基础。

这时候难免有善于思考的同学就来质疑我了,“你这不是自相矛盾?前脚让我们不要“伪勤奋”,现在又要摁着我头让我学,你把我当陀螺抽吗?”

这里我还是要说,对于很多事件我们不能仅仅停留在定性的角度,定量是十分重要的。我们现在要做一个严格的区分。

\begin{itemize}
    \item \textbf{伪勤奋}:这是我们在上一节批判的。比如一边抄单词一边想晚饭吃什么,或者看着书发呆一小时只翻了一页。这种状态下,你的大脑处于“待机模式”,产出几乎为0,纯粹是在感动自己。这种时间必须砍掉去休息。
    \item \textbf{低效积累}:这是我们现在要争取的。比如你今天累了,解不动压轴题了,但你还能机械地背背简单的知识,或者做几道简单的立体几何计算。这种时候大脑虽然不兴奋,但仍在“工作模式”。这种“中等效率”的时间,是构筑你高分基座的“钢筋混凝土”。
\end{itemize}

综上所述,“伪勤奋”是指用“疲惫状态”去假装做“高难度任务”(比如压轴题),结果什么都没干成。而我们提倡的“真努力”,是诚实地面对自己的状态:状态好就去突破自己,状态不好就去夯实地基。不要因为自己今天只有50\%的效率就感到焦虑或自责,只要你把这50\%用在了对的地方(比如背单词),这依然是宝贵的一部分。哪怕是爬,只要你在往前动,就比原地等待完美时机的人要快。

\section{自律与目标}
\begin{myquote}{《秦时明月》}
    当一个人的心中,有着更高的山峰想去攀登时,他就不会在意脚下的泥沼,他才能以最平静的方式去面对一般人难以承受的痛苦。
\end{myquote}

说实在的,目标其实才是最强大的驱动力。没有目标的人很难做到自律,或者说,他们不知道自律是为了什么。我认为首先应该有个长期(高中尺度)的目标,不用太具体也不要太模糊,比如可以是:以后从事什么行业;高考进多少名;考到什么学校这种。也尽量要有短期目标来有一定的动力,比如可以是下次考试考多少名。但这样的短期目标毕竟只是短期内以一个比较高效率的行动达到的,长期看来还是需要一个比较大一点的目标才行。

如果做到了有目标,就要尽量的对自己有一定的要求。我认为对自己要求高、要求严的人更容易进步,或者说目标其实也是一种“高要求”。

自律最基本的要求就是目标和上进心,只有有目标和上进心的人,切实希望进步、付出努力的人才有可能自律,才有可能进步。大概有以下几个要点。

\begin{itemize}
    \item 对自己诚实。欺骗老师和家长是很简单的事情,甚至欺骗同学都是不难的事情,但高考不会陪你演戏,你在假学习和科技上用的每一点小心思最后都会化成失败的泪水。
    \item 对于浪费的时间,要有负罪感,要知道自己的目标。
    \item 要有效率,要有目标感:我做的事情到底有没有对最后的结果有影响?而不是单单堆时间。
    \item 要把对自己的要求和目标尽量写下来放在能看到的位置上,不然在养成习惯之前就已经忘掉了。
\end{itemize}

% ==================================================
% 4. 人性的自留地 (The Human Side)
% ==================================================
\chapter{人性的自留地 (The Human Side)}

\section{青春期的荷尔蒙:关于恋爱}
——————如何在雷区跳探戈?

如果说你对高中有什么玫瑰金色的幻想的话,除了一帮好朋友在一起享受青春,就是在某个地方和你的男/女朋友亲嘴罢。但首先我要直接表达我的观点:非必要不恋爱;但如果你已经谈了,就要善始善终。

\textbf{1. 去污名化:你并没有“思想堕落”}

首先,我们需要通过一个生物学事实来赦免你的内疚感:在16-18岁,你的大脑边缘系统(负责情绪和冲动)已经发育成熟,而前额叶皮层(负责理性和控制)要到25岁才完全长好。对异性产生好感,渴望亲密接触,就像饿了想吃饭一样,是极其正常的生理反应。学校禁止早恋是基于管理成本的考量,而不是因为这在道德上是错误的。

\textbf{2. 现实主义评估:这是一项高风险投资虽然感情无罪,但作为理性的生存者,你必须明白当下的环境约束:}

\begin{itemize}
    \item 时间是零和游戏:你每天只有24小时——卿卿我我的时间=少学习的时间。
    \item 情绪波动是最大隐形成本:吵架一小时,通常需要后续三小时的心神不宁来平复。
    \item 你要面对的不仅仅是内部矛盾,还有老师、同学和家长。而家长和老师,在某种意义上才是你们这段感情最大的“敌人”
\end{itemize}

\textbf{3. 行动指南:“非必要不恋爱”}

\begin{itemize}
    \item 如果你单身(非必要不恋爱):\textbf{不要因为寂寞、跟风或为了证明魅力而去恋爱。}高中生活的容错率很低,如果没有遇到那个让你觉得“非他/她不可”的人,请保持单身。\textbf{孤独是增值的最佳时期。}
    \item 如果你已经谈了:就不要在“压抑”和“放纵”之间反复横跳,那样内耗更大。请尝试建立“战时同盟关系”:约定好,在教室、自习课互不打扰。把亲密限制在非学习时段。
\end{itemize}

不要神话”互相辅导”:除非你们水平相当且互补。大多数时候,最好的陪伴是“你在做数学,我在背单词,互不说话但知道你在”。

换句话说,你可能只是憧憬这种美好。时常看到有人在网上发“什么时候我才能谈一段甜甜的恋爱呀”云云,这种话证明你只是为了进入一段满足你预期的亲密关系,至于对方是谁,你对他真实的感情如何,都是“没那么所谓”的东西罢了。

谈恋爱没有你想象中的那么美好,等待你的大概率是矛盾、冲突、冷战和吵架;或者准备礼物、约出去玩、心情不好还要哄。你不能一边期待着恋爱中的亲密关系,一遍想念着单身时的自由。谈恋爱的确是一种难以忘怀的体验,但你在这上面分散的精力越多,你在学习上所花费的精力就越少。

回过头来,用人生的角度谈一谈。

从人生的角度来说,人生的主旋律应该是成长。你是有血有肉的人,没法在某一阶段被改造成学习机器。但因为高考对我们太重要了,我们把它放在很优先的地位。而恋爱,也是人生中非常重要的一部分,只是我们还都太年轻,在心里和生理上都不成熟,很容易产生一些错误的认知,还会影响学习,对人生产生重大影响的事情。

但一个人的本性就是容易被他人吸引。在一段漫长的人生中,大多数人在第一次不会遇到相伴一生的人。在没有接触谈恋爱的时候,一个人可能会对恋爱产生很多误解,比如觉得谈一次恋爱就要谈一辈子。恋爱对目前很多人都说陌生的话题,但它和其他的事情没有什么特殊性。谈恋爱其实和接触陌生的事物是一样的。不登高山,不知天之高也;不临深谿,不知地之厚也。

可能一个人在谈过恋爱之后,才能体会到,原来热切喜欢的那个人,也会形同陌路;不敢迈出那一步的人,原来是真的会错过那个同样期待着的ta。恋爱是人生中的必修课。

说了这么多,我的意见仍然没有改变,非必要不恋爱,但更重要的是,珍惜眼前人。

花有重开日,人无再少年。

\section{审美与教养}

\subsection{讲求“干净与卫生”的底层逻辑}

我们似乎已经默认了一个等式,高强度的学习=粗糙的生活,在高中或大学的日子里,大家对所谓“理科战神”的形象已经深入人心:“短头发,小胡茬,校服跑鞋、黑框眼镜”,但在这一节,我想请大家暂时放下手中的作业或习题,先来审视一下自己的“个人形象”与桌面卫生。
在物理中由熵增定律我们知道:
$$\Delta S \ge 0$$

这意味着,如果不对系统做功,事物的发展方向就是混乱,无序和衰败的。在生活中也就是你的头发,脸部会出油,书桌会自然的变乱。这是正常的,你不必苛责自己,在高考的高强度压力下,我们对抗生活熵增的能量确实会减少,所以我们可以把理科战神这样的形象抽象成由于能量输入不足而导致熵值过高的系统。

正如薛定谔所说:“生命以负熵为食”,我们就是在通过摄取能量来对抗熵增,维持系统的高度有序。而当我们如果允许自己出现蓬头垢面的形象,其实这也就是在变相的承认“我对这个身体系统的控制接近于摆烂了”,这种感觉会像蝴蝶效应一样,反馈到学习中,一个连自己身体都管理不好的人,很难让人相信他能管理好复杂的知识网络。

所以,我们呼吁:在时间允许、家庭条件允许的日子里,尽量管理一下个人卫生——这也
是以后集体生活的基石:包括但不限于洗头、洗脸、换衣服等。至于胡子,个人认为可以刮;
如果你固执的相信刮了会长的更多,那我建议你以少和家长起冲突以及个人感受为准。

\subsection{我们为什么要维护良好的学习环境}
首先,请允许我再次向大家展示耳熟能详的“破窗效应”,也就是如果一扇窗户破了没人修,很快其他的窗户也会被打破。环境暗示会诱导行为。

那么它如何应用到我们的学习中呢,我们来试着想象一个场景:当你正准备全力攻克一道圆锥曲线时,你面前的桌子左手放着你的笔记本,左上角有你吃完零食的包装袋,右手边放着自己的文件袋,你的手臂只能放在这些物品上来进行作答,手臂被硌的难受,这时候你觉着自己真的能做到全神贯注的解决你眼前的问题吗?

而造成你桌面如此混乱的这种原因,可能就是你不经意间随手放在一遍的食物袋子,又或者你没能及时的把那张试卷放到文件袋,然后破窗效益就产生了。

所以接下来我的建议,如果之前你对整理桌子觉着浪费精力,嗤之以鼻。那我劝你试一试再下定论,如果不行,你再换回你原来的习惯那我完全尊重你,但若你觉着收益大于投入,那么我建议你保持这样的习惯。

\subsection{关于“审美”}
在学校,你不妨可以看见身穿羊绒大衣,或者潮牌外套,鞋子的同学。他们衣品,个人形象确实比较不错。但这里我想说的对于高中生而言,这不是你该追求的审美。(对于“后高考时代”提升个人形象可详见5.4)

这里的审美,不是让你去追求名牌球鞋或违反校规的奇装异服,也不是世俗意义上欣赏梵高或毕加索的画作。真正的审美,是在千篇一律的校服,依然保持腰杆挺直的体态;是在潦草的草稿纸上,依然追求推导过程的逻辑美感;是在枯燥的刷题生活中,依然能因为一篇阅读理解或者一篇文字而引发你的思考与遐想。

作为读者的你,很有可能攻读数学,计算机等学科,你会发现,你暴力算出的解,与“能跑就行”的代码与真正优美的构造和优雅的代码是两码事。很多人会说这是“无用”的,这是没有意义的。但就像阿基米德希望在自己墓碑前留下最接近“圆”的几何,当你晚自习开始前眺望文化西路的晚霞,这一刻的颜色与光影,与你的排名,分数无关。但它就是“你还活着”,“这就是生活”的证明。

我们本章取名为人性的“自留地”,正是希望你可以有时间停下来读一本“闲书”,听一首“贝多芬”,这些没有功利目的的活动,恰恰才是纯粹的,也就是“无用之用”。正是这些“无用”,构成了你的精神屏障,防止你退化为只会做题的“单细胞”生物。

不要让自己退化成单一功能的做题机器。 即使身处“泥潭”,也要仰望“星空”;即使穿着校服,也要把自己收拾得干干净净。这是一种不向环境妥协的、作为“人”该有的品质。
% ==================================================
% 5. 抗压能力与心态
% ==================================================
\chapter{心态与心理健康}

\section{高中整体心态}
\begin{myquote}{罗曼·罗兰}
    世界上只有一种真正的英雄主义,那就是在认清生活的真相后依然热爱生活。 
\end{myquote}
\subsection{适应力超乎你的想象}
在刚上高中的时候,由于初中还没有晚自习,我很难适应持续三四个小时、中间只有十分钟的晚自习;但在一段时间之后,很好地适应了这样的生活,甚至在下了晚自习之后还有多余的精力在家里稍微看看书复习一下。我的意思是,人的适应能力是很强大的,你要相信你会适应这一切,你有能力应付这一切。

有句话叫“相信相信的力量”,这句话很像“真实自有万钧之力”,初听莫名其妙,细品回味悠长。在高中的三年你会成长,会失败,会喜悦会落泪,会在长夜中哭到缺氧、难以入眠,也会在夕阳下快乐的和朋友嬉戏,但一定不要忘了自己是谁,不要忘了相信自己。你可能会被打击到差点一蹶不振,看着光鲜亮丽的大家自惭形秽,但也不要忘了相信的力量。痛苦,往往也就意味着你在成长。那些痛苦的日子会变成养分,供给起一个更加成熟、内核更加坚固的自己。所以,相信自己,体味这个过程。王尔德在《道林格雷的画像》中写到:你拥有青春的时候,就要感受它。不要虚掷你的黄金时代,不要去倾听枯燥乏味的东西,不要设法挽留无望的失败,不要把你的生命献给无知、平庸和低俗。这些都是我们时代病态的目标,虚假的理想。活着!把你宝贵的内在生命活出来。什么都别错过。我将这句话当作自己的座右铭,同时也送给各位。

\subsection{焦虑}
一个常见的心态是,过度关心成绩,过度焦虑高考。从我意识到我的人生中一定会有高考这件事情之后,一想到它我就会感到崩溃:百万人级别的考试,“一考定终身”,甚至无法相信我竟然要去面对高考。坏消息是,很难避免焦虑,人都会焦虑,只是在面对它的方法和措施上有差别。

首先要认识它:为什么焦虑?可能是因为我能力有限,做题总是错,注意力也集中不起来,觉得自己很失败。你认识了它,知道它是什么,你才有可能战胜它。要清晰的描绘和剖析自己的内心,而不是用一个抽象的状态来掩盖真正的原因。

怎么面对呢?最有力的其实是时间。因为人是需要时间成长的,前人留下的经验和诗句你在当初看的时候只是觉得言之有理,但只有真正经历过才会明白是什么样子。你在高三的时候回头看刚上高中时的不适应和焦虑,大概率连具体的想法和心态都忘记了,只是记得当时很难过。泰戈尔说,未曾哭过长夜的人不足以语人生,哪怕你懂得了很多道理,明白“不能焦虑”,但在当下的你做不到也是很正常的,你需要在辗转反侧的夜晚和痛苦的经历中成长,这是无可奈何的事情。短期呢?我认为可以行动起来,做作业、预习复习或者锻炼,把自己做的事情记录下来,等到回头看自己做了这么多事情,就不会感到很焦虑了;或是写写日记,不用长篇大论,只是把内心的烦恼明确、具体地写出来,就已经在减轻你的负担了。不要为已经浪费的时间和日子哭泣,抬头看看还剩多少日子,从下定决心不浪费时间的时候开始也不晚。

对于焦虑,接受它。每个人都受过这样的痛苦,你不会一直这样,它也不会在长时间内影响到你的结果。人生是一场漫长的均值回归,不论你在短期内,乃至一两年、三四年内怎样,只要一直付出努力,你最终会向“真正的你”去靠拢。短期内可以通过做各种事情去缓解。通过不断地与焦虑抗争,哪怕每次都没有摆脱它,你也是在切实地慢慢进步,最终会在某个云淡风轻的日常回忆起以前的事,发现自己已经走了这么远的路,已经不是原来的自己了。

\subsection{“无所谓”}
第二种常见的心态就是过度无所谓:考得好也好,考的差也罢,日子不都得过吗?这样的心态更适合于事后而非事前:高中时期,还是得努力争取一下。只是争取的时候不用强求,而是以一种:“我尽力了,我的能力就这么多,我上去了是很好,没上去也是能力到这了,我没有遗憾”的心态,只要尽自己的努力,结果没办法控制,最后只是要给自己一个交代罢了。不论高考怎么样,这三年没有虚度,这就是最大的成长了。

\section{高中日常心态}

\subsection{成绩其实是附带品}
说实话,之前听别人说“做好自己的事,成绩只是提升过程中的附带品”的时候,我感觉就是扯淡。但现在看来,过度关注成绩没有什么好处,但漠不关心也不是一个健康的心态。一个比较健康的心态其实是“但求耕耘”,在结果上“问心无愧”:在日常学习的过程中,对自己的听讲、知识点、错题认真负责,争取把每一部分都弄会,在做题上也是尽力而为 这样自然地你的成绩也不会差。当然这样说也有点不靠谱,但最好的方式确实是投时间在平常,而不是考前疯狂突击你的那错题本然后祈祷考个好成绩。你不可能指望在平常放纵,到了考试前一晚变成什么都会的超人。现实一点,不要幻想。

\subsection{好好对自己}
在高中不意味着你是个学习机器。我们不要过度重视理性而忽视感性。很多事情都是很日常的小事,只是在高考面前没有办法。比如,你喜欢吃小蛋糕,吃甜品,那考完就奖励自己吃一次或有些进步了就吃一次,珍惜这些日常中的小确幸,珍惜自己内心中那块柔软的地方。很多人都容易被压力摧垮,就是因为他们的生活中把学习放在了绝对的大头。要在生活中多一些小事,多一些小小的成就感,小小的陪伴,小小的开心和幸福,这样你的内心才会稳固,不会被外界的风雨催伤。

所以,偶尔花时间在这些“没意义”的小事——看看小说,撸撸猫,弹弹琴,散散步,吃点好吃的——上,他们附带的对你的精神和心态上的好处,远比你把这些画在学习和打游戏上有意义。

\subsection{挫败感和自卑}
除去少部分家境优越和天赋很好的同学外,很少能有人没有自卑过。自卑和感到自己失败是两种很有害的心态,也很难一次性就克服,需要漫长的时间和自己和解。

固然,每个人在自己的生活中都是主角,但其实在他人的生活中都只是NPC罢了:世界不会像小说中那样围着你转,你只是一个普通人。没有像明星那样优秀的外表,没有像演奏家那样高超的天赋,没有像职业选手一样的反应能力... 我们要承认,自己是普通人,能力有限,天赋有限,但这并不意味着承认自己平庸。平凡不是平庸,而是一种认清现实后还愿意努力奋斗的乐观态度。

我们可以在心情还不错的时候多记录一下自己的感受和自己的优点,把自己的感受和事情的成败剥离开,做到在成功的时候也没那么喜悦,这样在失败的时候也不会太挫败;或者在心情不好的时候,拿出来看看,发现自己还有很多值得喜悦的优点:哪怕不擅长,热爱也足以。

自卑其实是一种过高要求的体现,没必要对自己太过于苛责。成长到今天,每个人都不容易,大家都有各自的优点和缺点,没必要拿自己的短板去和别人的长板作比。接纳自己是一个要用一辈子去实践的话题,尽力后交给时间就好。

\subsection{平静而有力的,接纳一切}
不知道大家有没有在入睡前突然想起前阵子的自己做的一切糗事而感到无地自容,或是看到班上的“嘉豪”而开怀大笑。我想说的是,人的成长需要时间,请多给自己和他人一些时间吧。

或许有些人在生活中的言行举止让你感觉怪异而难受:彰显自己独特品味的小众哥、没实力硬要耍帅的“嘉豪”“嘉欣”;上台在 seewo 白板上看股票、打cmd的“乐子”,我想说,很多事情的错并不是他们个人。

我们的教育在自我认同形成的过程上不仅帮不上忙,反而是一味地打压,导致很多人在这方面有了一生的不断地执着追求。那些酒桌上光着膀子吹牛的油腻大叔,现在看起来油腻而面目可憎,但他们大概率也只是自我构建的失败者,才会“走过半生,归来仍是少年”:执着的寻求着他人的认可;嘉豪不也是吗?他自己没有什么错,只是想在被打压的过程中,展示自己的魅力和能力;只不过以一个不合适的方式进行了。说到底,他没有做错什么;反而是那些以此为由进行嘲笑霸凌的人,性质说不定更恶劣一些。

很多这样的同学没什么恶意,只是想展示自己,这无可厚非;方式错了,教育一度就好。心灵坏了,怎么教育也没有意义了。我呼吁,给大家多一些时间,不要以自己的低俗和无知作为品德来展示,不要把对弱者的歧视当作理所当然:睁开眼睛看看你们自己的嘴脸吧,这是你受到的教育所要培养的人吗?

当然,这不尽然是任何人的错,人是需要时间成长的,我只是想,在大家成长的路上,对自己和对他人都是,多一些耐心,少一些嘲笑。多一些帮助,少一些歧视。很多事情回头看才知道是怎么回事,但对他人的伤害已经形成,无法回头了。

现在的社会上戾气很重,身为高中生,你不能一边念叨着教育体系中的人文关怀的缺失,一边做着恶劣的行为。我希望大家都能好好的,只是向上走,不必听自暴自弃者流的话。能做事的做事,能发声的发声。有一分热,发一分光,就令萤火一般,也可以在黑暗里发一点光,不必等候炬火。此后如竟没有炬火:我便是唯一的光。

\section{虚无和实感}
\begin{myquote}{格里高利·罗伯兹《项塔兰》}
    有时候必须要用爱微小的事物,才能在这个充满了庞大痛苦的世界里坚持下去。
\end{myquote}
\subsection{悬浮在真空中的“生物模型算法”}

你是否经历过这样的时刻:明明坐在教室里,周围人声鼎沸,你却觉得一切都像隔着一层玻璃,声音很远,画面很失真?你看着满卷子的红叉或者对勾,内心毫无波澜,既不难过也不开心,只觉得“没劲”。感觉,活着好像也挺好的,但是死了也没什么所谓;反正还是先活着吧。总之,你找不到活着的乐趣,但又有些恐惧死亡,于是在这种恐惧的驱动下,做着这些“毫无意义”的事情。

这是一种比焦虑更可怕的症状——虚无感。

在长达三年的备考中,我们很容易陷入一种误区:认为只有高考结束后的生活才是“生活”,而现在这三年只是一个必须忍受的“过程”。当我们把当下的每一天都仅仅视为通向未来的“手段”时,我们就把自己当成了工具。

马克思把这称为“\textbf{异化}”。而对于我们来说,这种感觉就是“\textbf{悬浮}”。你感觉自己不是一个有血有肉的人,而是一个为了提升排名参数而存在的生物算法模型。当你切断了与当下的感知,你的心就会慢慢变空。这就是为什么很多“做题家”赢了高考,却在大学里陷入了更深的抑郁——因为他们早就丧失了感受“活着”的能力。

\subsection{重建“实感”:对抗虚无的锚点}

如果说焦虑是对未来的恐惧,那么虚无就是对现在的麻木。对抗虚无的唯一解药,不是更高远的目标,而是粗粝的、具体的实感。

如果你觉得飘在了半空,请试着抓住宅些沉甸甸的“锚点”把自己拉回地面:
肉体的痛觉与触觉——不要鄙视身体。当你觉得大脑在空转时,去操场狂奔两圈,拿冷水洗把脸,去感受肌肉的酸痛、肺部的灼烧感或是冷风刮过脸颊的刺痛。身体的实感是确认“我存在”的最直接证据。不要让你的身体仅仅沦为承载大脑的容器,去使用它,去感受它。

具体的创造,而非分数的积累: 做题是一种“由于他人评价”而产生的价值,它很容易让你觉得虚无。你需要做一些“\textbf{为了自己且有实体产出}”的事。

并不是让你去搞大发明,而是指:整理乱成一团的课桌,收拾的整整齐齐;在草稿纸的角落画一只小猫o(=•ェ•=)m;写一段只给自己看的日记。

这些微小的、可控的、有具体反馈的行动,能让你重新掌握对生活的控制权。或者说,这是一些小而具体、可以摸得到碰得到的幸福。

从“宏大叙事”回归“具体的人”: 陀思妥耶夫斯基说过:“爱具体的人,不要爱抽象的人。” 当我们把周围的同学看作“竞争对手”或“全省排名分母”时,我们是孤独且虚无的。试着去关注具体的人:同桌今天换了新眼镜,后桌感冒了在咳嗽,晚霞照在隔壁班同学的脸上。 去建立真实的链接。 一句真诚的“这个题我也不会,太难了”比互相假装淡定更有力量。虚无感最怕的就是这种带着体温的、具体的真实互动。

最后记得,相信自己,你只是在成长的路上碰到了一些小小的挫折,不要让他们打败你,相信自己能够战胜这些,能够以一个胜利者的姿态,平静温和而有力地走在你的路上。

\subsection{警惕“电子止痛药”}

最后,一个严肃的提醒。

当虚无感袭来时,最容易诱惑我们的就是手机。短视频、爽文、游戏,它们提供的是一种“\textbf{伪实感}”。它们用高频的刺激来填补你内心的空洞,让你误以为自己充实了。

但请注意,这种填补是麻醉剂。当你放下手机,那种空虚感会以十倍的重量反扑回来,让你觉得更加虚幻和荒诞。

真正的休息是找回感知,而不是屏蔽感知。

宁可在发呆中观察窗外的树叶怎么飘落,也不要让算法接管你的大脑。允许自己感到无聊,允许自己发呆,因为正是在这些所谓的“空白”里,你的灵魂才有机会喘息,重新附着回你的躯体之上。

活着,不仅仅是为了那张录取通知书,更是为了此时此刻,你能感觉到笔尖摩擦纸张的阻力,能感觉到心脏在胸腔里的跳动。

这也是你作为“人”的证据。

没有人是一座孤岛。
\chapter{高考之后}
\section{升学方式与经验分享}
在山东主要有这几种升学方式(按可能性排序):
\begin{enumerate}
    \item 高考
    \item 综评
    \item 强基
    \item 竞赛
\end{enumerate}
竞赛相关话题此处不做讨论,高考的内容前文所述已经差不多完备,接下来简单分享一下综评和强基的经验。

\subsection{综合评价招生}
综合评价招生,简称“综评”,和强基一样,都是让你展现一些特长和高考之外的品格以换得较低分录取的方式。

主要依据:高考成绩、高中学业水平考试成绩和学校考核。三者不同学校有不同比例\footnote{如山东大学在2025年采取综合成绩=高考总成绩(换算成百分制)×0.85+学校考核成绩×0.15},其中最主要的是高考成绩,其次是高中成绩和学校审核。

\footnote{山东省教育厅发布的《关于做好2025年普通本科高校综合评价招生试点工作的通知》}2025年,山东省在山东大学、中国海洋大学、中国石油大学(华东)、哈尔滨工业大学(威海)、青岛大学、山东师范大学、山东科技大学、青岛科技大学、山东财经大学、北京外国语大学、中国科学院大学、南方科技大学、上海纽约大学、昆山杜克大学、深圳北理莫斯科大学、香港中文大学(深圳)、上海科技大学、浙江大学、华南理工大学19所高校开展本科综合评价招生试点。

具体来说,综评的流程大致如下:
\begin{enumerate}
    \item 3-4月 各高校招生简章发布
    \item 4-5月 网上报名
    \item 5月底6月初 高校公布初审名单
    \item 6月20日前  高校复试
    \item 6月25日前  公布复试名单
\end{enumerate}

以下以中国科学院大学和哈工大(威海)为例,简单介绍一下综评的流程和经验。(后者仅介绍面试)

\textbf{1.初筛}~ 初筛是机器筛选,主要看高中课内成绩,高三在几次大考中要至少过一定分数线线一定次数。(具体分数已不记得)

\textbf{2.复试}~ 复试是人工进行。面试专家团队,在开展综合评价选拔各省市(山东是青岛西海岸)的指定考点进行面试。各省市面试专家团队在面试前随机抽签组成面试专家组,每个面试小组由3名左右专家组成。获得面试资格的考生于面试前随机抽取分组编号,每5名左右考生为一组共同参加面试,每组面试时间为1小时左右。

问一些比较有趣的问题:比如“外星人来了怎么交流”“列举生活中的黄金分割”(当然不是直接问)之类的。总之是既考验交流能力,也考验知识的积累。由于是抢答制,所以思考的时间和抢答的速度要做权衡。

最重要的是,\textbf{自信}。面试官并不要求你回答的多么完美,而是希望看到你积极思考和表达的态度。不要害怕回答错问题,勇敢地展示你的思维过程和个性。不要低个头,要勇于进行眼神接触。说话的时候不卑不亢,充满逻辑思考。

一个有用的小trick是\textbf{态度}。可以把你的材料提前彩打出来,塑封一遍,面试的时候拿出来给面试官看。这样会让面试官觉得你很重视这次面试,态度端正。

剩下的就是一些体测,不提也罢。

至于哈威的面试,更加简陋了。大概就是初筛过了之后,去哈威校区的一个教室坐着等;到了你要抽题目,抽一个现成准备好的,比如“新时代国防”,给你几分钟写一写画一画,就进去表达。说完了之后专家们再根据你的个人情况进行一些提问。不过有些抽象的是,还会问一些什么“你对哈威有多少了解?”“校训是什么”“今年是建校多少周年”这种抽象问题。更抽象的是要强调是“校区”不是“分校”云云。

\subsection{强基计划}
强基计划是国家为了选拔有志于基础学科研究的拔尖创新人才而设立的一种招生方式。主要依据:高考成绩和学校考核。高考成绩占比70\%,学校考核占比30\%。

理论上,强基计划适用于数学、物理、化学、生物、历史、哲学等基础学科,但实际上很多工科都混进去了,比如计算机、电子信息等。

笔者以参与过的中科大强基计划为例,简单介绍一下强基的流程和经验。

\begin{enumerate}
    \item 4-5月,简章公布,网上报名
    \item 高考后、7月4日前各省(区、市)提供高考成绩高校确定考核名单并组织考核
\end{enumerate}

但这并不意味着高考分数出来前就会出面试分:这意味着学校可能会“看人下菜碟”:看你高考分考的怎么样来给面试分,而与具体的面试无关。

总之,笔试各个学校的难度不一样。中科大的笔试相对简答,考数学和物理,共200分,一百分出头一点就过了。

面试也就是走个流程,最后还是高考分说了算。

但要慎重的一个点是不能转专业,所以如果你对专业不确定,建议不要贸然报考强基。也没法保研到其他的专业,只能考研。

\section{“后高考时代”的心态调整}
\begin{myquote}{萧伯纳}
    生活不是为了发现你自己,生活是为了创造你自己。
\end{myquote}
读到这里的你,可能已经结束了人生中最大的一场考试或者现在的你相信一定憧憬过高考结束之后会怎样怎样。当你结束了高考以及一系列“综评”与强基之后,现在的你除了有对与高考成绩与录取院校不确定的紧张之外,这里作为过来人的笔者,还是希望跟你聊一聊你可以精进或者改变的地方。
\subsection{接受现实,转变思维}
首先很不幸的一点,我必须向你承认,高考结束后光靠你努力做好的事情已经越来越少了,很多时候努力仅仅只是一个必要条件,这是我们走出“小镇”的资本,也是你未来想要做出你想做成的事的必然。可能你在高中已经觉着很多时候努力比不上同桌轻松写意的攻克压轴题。但在大学,这里的一切都会“变本加厉”,你身边的同学可能坐着的是cmo(全国高中生数学联赛)银牌,另外一个同学早早学完你正在学习的课程,取得的成绩令你难以望其项背;又或者朋友“全力依父”带上了你梦寐以求的劳力士手表,早早把雅思托福等语言成绩考出随时准备出国或借助家里的资源。可是你似乎在你的家乡很少有听过相关的信息。更有甚者,你发现大学的小组合作认真办事的只有你或者少数几个同学,大部分同学坐享其成,你会怀疑他们怎么考上的你这所大学。

看到这里,你可能会感到无语\sout{甚至难绷},为什么我努力了前18年最后跟我\sout{说这拉了坨大的}说现实竟然是这样的。我现在很想知道作为读者的你觉着我该如何给你建议。因为现实不是爽文,个人的发展很大程度上是受大环境和小环境(家庭、学校)和身边人的影响的。你可能觉得不公平,但现实就是这样。

我想说的是,我以上说的这一切,其实完全是在高中思维下所构建的,也就是与他人的比较,仅仅只是单一的评价标准,要么是成绩,要么是家境。在大学“大人,时代变了”:一切的评价体系都改变了,就像五子棋向围棋的转变。你感到痛苦是正常的,因为你开始步入社会了,你需要意识到,你的成绩,这一单一的指标不是社会的评价标准,要社会财富真是完全按成绩分配的,那有钱人都必须个个是清北本科常春藤名校博士,事实很明显不是这样嘛。

所以,我们需要转变的是我们的思维,升级我们的“努力”,而非还和高考前那样做题,高考前我们可以说“忍耐”“想得开,挺得住”,但高考后,我们的努力,应该去为我们的未来寻找可能,去向自己想要成为的人或想要学习的事去靠拢。

那些家庭优越,能给他们带来大量普通人难以接触到信息的人或者天赋异禀者确实在18岁这一节点上赢在了起跑线,我们必须大大方方的承认这就是他人的优势,然后利用好你自己的可能是在“小镇”培养的坚毅的品质,放下比较,争第一的心,去做好自己的事就足够了。

\subsection{专注自己,打破信息壁垒}
要做到这一点,我有两点建议:

第一不要强行定义与比较:也就是不要定义什么是优秀的,你在成绩上打败了他,你就是胜利者,你是优秀的,他是平庸的;或者他的体脂率,颜值高于你,他就完全胜利了,把你完全比下去了。这样的比较说明不了什么,能给你带来“赢”的爽感只会不断麻木你,让你离自己的目标越来越远。作为生长在完全不同家庭,省份,接受不同学校老师的教育与不同的学生相处,这这么多变量没控制,你凭什么就能下结论?这里再重复一次,专注自己,不要陷入比较的怪圈。

第二,打破信息壁垒,主动求索。家里没人规划你该不该研究生出去留学,如何准备语言,或者如何做好科研。但在现如今互联网与ai极速发展的时代,你是否会能登陆谷歌浏览器,是否会使用ChatGPT,Gemini等顶尖工具,是否在计算机学习中可以在GitHub上学习他人的代码是如何写出来的。以及你能否主动发邮件去约教授,老师去问出心中的疑惑。

所以这一节最重要的就是要告诉你,不要执着于在他人的赛道进行比赛了,你要做的就是设计你心中最适合你的轨道与成长路径。

\begin{note}
    我们在这里不做“大学生活祛魅”的部分:因为没必要,等你上了大学再看也不迟。而且生活本来就是这样,是你赋予了过多的意义而不是它本身存在这么多意义。所谓祛魅其实也算是去除了你对它的希冀与盼望。我们希望以温和的语调为你带来希望,而不是绝望。生活的意义在你,不在其他。

    要记住,你远大于你的大学。
\end{note}


\chapter{拾遗与杂谈}
这部分简单谈谈备考的心态、常见的疑问和一些杂谈。
\section{高三考前心态}
\begin{myquote}{史铁生}
    命定的局限尽可永在,不屈的挑战却不可须臾或缺。
\end{myquote}
\begin{questionbox}{Q:~高考前焦虑}
    高三学的时候总是不确定学的东西会不会考,现在看的东西到底有意义吗?
\end{questionbox}

\begin{answer}{}
    这是一个很常见的想法。但没有人能够预知未来,这样的焦虑其实只是你对高考和自身知识掌握度的怀疑。只有你在考完回看的时候,才会发现这些记忆里的点连成了线。

    所以我的回答是,在现在做的一切事情都是有意义的,但只有回头看才会发现它们的意义。不要过度焦虑未来,踏实地做好现在的每一件事就好。
\end{answer}

\begin{questionbox}{Q:~考完一模/二模就大局已定了吗?}
    我一模/二模考得很差,感觉高考也没什么希望了,是不是就放弃吧?
    
\end{questionbox}

\begin{answer}
    绝对不是这样。说白了,平常考的什么样都无所谓,只有最后一场考试是决定性的。高考没有平时分,不需要在意平时考试的分数高低。
    
    真正值得你在意的是自己的考试状态、心态和不懂、不熟练的知识点。
\end{answer}

\begin{questionbox}
    {Q:~我没有目标,没有梦想,没有动力,甚至到了厌学和抑郁的地步,该怎么办?}
\end{questionbox}

\begin{answer}
    第一建议是看医生。如果你觉得自己已经到了厌学的地步,甚至对生活都提不起兴趣了,建议你去看心理医生。高考固然重要,但你的精神健康更重要。

    活着的意义就在于活着本身。存在本身就是对抗荒谬最有力的武器。

    如果你觉得活着已经很累了,甚至有些抑郁的情绪,请直接去看医生。我们什么都可以放弃,除了生命。

    如果没有这么严重,那我建议你从小目标开始积累成就感和幸福感,慢慢走下去,总会找到答案的。你最需要的是时间而不是对你的规训。
\end{answer}
\begin{questionbox}
    {Q:~高考之前我是不是什么都掌握了,胸有成竹的上去考?}
\end{questionbox}

\begin{answer}
    显然不是。你会发现不会的还是很多,但还是要去考。所以和你的月考、模考的心态是很类似的。
\end{answer}
\newpage
\section{珍惜时间}

\begin{myquote}{王尔德}
    你拥有青春的时候,就要感受它。不要虚掷你的黄金时代,不要去倾听枯燥乏味的东西,不要设法挽留无望的失败,不要把你的生命献给无知、平庸和低俗。这些都是我们时代病态的目标,虚假的理想。活着!把你宝贵的内在生命活出来。什么都别错过。
\end{myquote}

朋友们,王羲之在一千六百年前的《兰亭集序》中写道:“虽世殊事异,所以兴怀,其致一也。后世览者,亦将有感于斯文。”的确是这样。

如果人是像很多自然规律那样遵循本福特定律,以对数\footnote{灵感来源于毕导视频《这个定律,预言了你的人生进度条》}的方式感知世界,假设你能活到80岁;假设四岁开始有记忆,那么你的生命的中点,其实是十八岁。也就是说,你现在正处于你生命的黄金时代。

现在是你的生命力和创造力的巅峰,是你的全部精神活力都在迸发的年纪。请珍惜它,当你能够呼吸到新鲜空气,在晚自习下课能看到炫彩的晚霞,在海边散步能够观澜听涛,在操场上奔跑能感受到风的脉搏的时候,尽情地去做吧!

你现在经历的事情不会再来一遍了。你在体育课上摔的跤,在课上回答不出来问题的尴尬,在多年后只是成了回忆的锚点。

趁你还年轻,珍惜眼前人,珍惜眼前事,珍惜眼前景。

过去我总是听所谓\textbf{“人无法同时拥有青春和对青春的感受”},觉得现在已经在体味青春的美好了;但回过头来发现,已经不会有那么多人,聚集在一个小教室里,共同度过最困难的三年了。不会再有这么放松和肆意的日子了。

\section{自我剖析与成长}
\begin{myquote}{卡耐基}
    人类最大的痛苦,莫过于自我剖析。
\end{myquote}
\subsection{像手术刀一样解剖自己}

如果在高中这三年,你只学会了做题,那太无聊了。因为这三年高压环境,恰恰是观察自己人性的最佳实验室。

什么叫自我剖析?不是在日记里无病呻吟,而是对自己极度诚实,诚实到近乎残酷。我们要手术刀式的解剖自己的情绪,看看里面藏的东西是什么。

举个例子:考试没考好,你的第一反应是什么?可能是:
\begin{enumerate}
    \item “老师出题太偏了,运气不好。”(借口)
    \item “我其实复习了,但没考好,不仅丢人,还对不起爸妈。”(恐惧)
    \item “中档题做的不全、错题本只是抄上题没看、简单题计算错误频出”(真相)
\end{enumerate}
只有当你剖析到第三层时,成长才真正发生。前文提到的拒绝“伪勤奋”,本质上就是一种自我剖析。成长就是无数次地发现自己卑劣、懒惰、虚荣的瞬间,然后没有逃避,而是选择原谅并修正它。刚开始很难承认,自己是一个贪婪、懒惰、虚荣的人。但这是成长的必经之路。

不要害怕承认自己“嫉妒”同桌,不要害怕承认自己“想偷懒”。承认它们,它们就只是你的一部分;否认它们,它们就会成为你的主人。

当然,不止于学习。当你感到愤怒、焦虑、孤独时,试着去剖析它们的根源。你会发现,很多情绪的背后,其实是对“被接纳”“被认可”“被理解”的渴望。当你真正能够慢慢接纳自己,接纳这些情绪的时候,你就会发现,自己变得更强大了。
\subsection{破碎后的重组}

很多同学在初中是“天之骄子”,到了高中会经历一次“信仰崩塌”——发现自己再怎么努力也只是普通人。这很痛,但这正是成长的契机。

海明威说:“\textbf{生活总是让我们遍体鳞伤,但到后来,那些受伤的地方一定会变成我们最强壮的地方。}”

如果以前的你的自尊心很强,但建立在“我比别人更聪明、更优秀”这样的幻觉之上,那么你很容易因为月考排名的下滑被击碎自信,也无法正视自己的缺点。

真正的自己,自尊心应该建立在对自己的高标准严要求上,而不是对他人的攀比。你要相信,\textbf{成长是一个长期的过程},你不需要在短期内证明自己比别人优秀。

所谓的成长,不是如果你变成了一块无坚不摧的石头;而是你变成了一汪水。你依然会感到痛苦、焦虑和迷茫,但你不再会因为这些情绪而停滞不前。你知道今天崩溃大哭一场,明天依然可以平静地前进。

\subsection{种树的最佳时间}

最后,关于成长的速度。

我们习惯了短视频的即时反馈,习惯了游戏里的“点击即送屠龙刀”,但在自我成长这件事上,没有倍速播放。

你可能坚持了三个月的早起,成绩依然没有起色;你可能剖析了自己的性格缺陷,第二天依然犯了同样的错。请不要苛责自己。人的改变是按“年”为单位计算的。我们前面说过,人生是一场漫长的均值回归。

不要急着向世界证明你“变了”或“成熟了”。真正的成熟是静水流深。

如果此刻你觉得自己一无是处,浑身毛病,甚至有点讨厌自己。没关系,看见黑暗,是变好的开始。 给自己一点时间,允许自己像植物一样,在看不见的泥土里,慢慢扎根。

当你的积累足够多的时候,你会逐渐平静下来,发现自己已经变成了一个更好的自己。

\section{浅谈个性}

\begin{myquote}{荣格}
    你所抗拒的,终将会以你无法抗拒的方式出现。
\end{myquote}

\subsection{“你真懂事”是赞美吗?} 

从小到大,你应该没少听过这句话:“这孩子真乖,真懂事。” 在很长一段时间里,这被我们视为最高的勋章。为了这枚勋章,我们学会了察言观色,学会了压抑自己的欲望,学会了把家长的期待当作自己的志愿。我们像一棵棵被精心修剪的盆景,被砍掉了旁逸斜出的枝丫,只为了长成大人们审美中标准的样子。

《士兵突击》里,高城说过:\textbf{越早熟的人,往往也就越晚熟。} 这句话的意思是,过早地“懂事”,往往意味着过早地失去了自己;在未来的日子里,要花更多的时间找到自己是谁。

但我想告诉你一个残酷的真相:在工业化教育流水线上,“懂事”意味着“好管理”,意味着你是一个合格的标准化零件。但在真实的世界里,这往往意味着“缺乏创造力”和“自我面目的模糊”。

如果你在18岁之前,从来没有过一次真正意义上的“叛逆”,从来没有在这个系统里发出过属于自己的杂音,那你可能需要警惕:你真的知道自己想要什么吗?

\subsection{找回你的“野性”}

我们这代中国学生,最大的不足往往不是智力上的,而是\textbf{野性},或者说,某种程度上的\textbf{勇气}。

我们太习惯于等待指令了。没有作业就不会安排时间,没有考纲就不会学习,没有排名就不知道自己处于什么位置。我们活在他人构建的坐标系里,一旦把你扔到没有坐标系的旷野里,你就会开始迷失。

这种“野性”并不是让你去打架违纪,而是指:

\begin{enumerate}
    \item 敢于拥有不被认可的梦想: 哪怕全世界都让你学金融计算机,你依然敢在周末偷偷写你的小说,研究你的昆虫。
    \item 敢于提出不同的声音: 哪怕老师在台上讲得头头是道,你依然敢在心里(甚至举手)质疑他的逻辑。
    \item 敢于让自己“不标准”: 你不需要让每个人都喜欢你,你不需要在那张雷达图上每一项都拿满分。做一个有棱角、有缺陷、但鲜活的人,远比做一个完美的塑料模特要迷人得多。
\end{enumerate}

\subsection{你的“出厂设置”}

在高中这个高压锅里,我们为了生存,不得不戴上统一的面具。这没关系,这是战术层面的妥协。 但请你务必在心里留一块自留地,把那些奇怪的想法、看起来无用的爱好、那些不被允许的情绪,都好好地藏在那里。

不要让高考这个模具,彻底改变了你的形状。

当你走出考场的那一刻,我希望你不仅仅是拿到了分数的赢家,更是一个依然记得自己名字、依然有棱有角的少年。

\section{情绪阉割:“情绪稳定”不一定是好事}
\begin{myquote}{塞林格}
    生活中最重要的事情之一,就是学会感受痛苦而不是逃避它。
\end{myquote}
\subsection{系统里的“静音键”}

在我们的成长教育里,有一套隐形的“情绪羞耻机制”。 每当你考砸了想哭时,总有人告诉你:“哭有什么用?哭能把分哭回来吗?” 每当你因为压力大而愤怒时,总有人劝你:“要成熟一点,控制好自己的情绪。” 甚至当你考得好想大笑时,也会有声音提醒你:“要谦虚,不要得意忘形。”

久而久之,我们学会了一项生存技能:给自己按下“静音键”。我们将愤怒吞进肚子,将眼泪憋回眼眶,在这个充满竞争的教室里,我们努力把自己修练成一尊“泰山崩于前而色不变”的石佛。\sout{怎么的哭都不行啊}

我们误以为这就叫“心态好”,这就叫“成熟”。 但对不起,那不叫心态好,那叫麻木。那不叫成熟,那叫“\textbf{情绪阉割}”。

高考这台机器,最喜欢的燃料是冷静、理性、不知疲倦的执行力。为了迎合它,我们试图切除自己“人性”中那些波动的、不可控的部分。但请你记住:你是一个人,不是一个只会做题的AI。是人,就会痛,就会累,就会有想砸东西的冲动。

\subsection{失去痛感,也就失去了快乐}

“情绪阉割”最大的副作用,不仅仅是让你变得压抑,而是让你丧失了感受美好的能力。 人的感知系统是连通的。当你为了防御痛苦,给心灵打了一针厚厚的麻药时,这针麻药同时也屏蔽了快乐。

我们前面简单谈过关于“虚无感”的议题。我个人认为,过于“早熟”和被要求“情绪稳定”的同学,他们不是没有情绪,而是一直被压抑。如果他们做到了稳定的情绪,大概率是因为根本不在乎这一切了,因而产生了麻木感。

当然,这和我们前面说的“平静的面对”并不冲突。真正的平静,是在经历了痛苦之后,依然能够坦然地接受它,而不是压抑它。我们不否认痛苦的存在,但我们拒绝让痛苦成为阻碍我们前进的理由。你可以哭可以愤怒,但哭完愤怒完之后,你依然要继续前进:这才是我们的意思。

\subsection{请允许自己“发疯”}

在这本指北里,我最想给你的建议之一就是:不要试图做一个每时每刻都情绪稳定的“正常人”。

如果你感到痛苦,请不要用“大家都很难,我这点苦算什么”来自我PUA。你的痛苦是真实的,是值得被尊重的。痛苦是正常的,非得要去比谁更痛苦的是不正常的。

在我们的文化里,男性尤其被要求坚强和理性,所谓“男子汉大丈夫”“男儿有泪不轻弹”云云,受到打击更是被要求“坚强”。这种文化是有毒的,它是反人性的。真正的坚强,不是连情绪本身都不允许,而是勇敢的承认、面对,积极的战胜它。所以男生也可以哭泣,也可以释放自己,不要等到一切都来不及了,才后悔。

同样地,女同学被要求“温柔”“体贴”,被要求“情绪稳定”,这也是一种枷锁。你完全可以有自己的情绪波动,不需要为了迎合他人的期待而压抑自己。你可以强大,可以理智,也可以温柔体贴,这都是自己的选择。我们反对的,是连选择的机会都没有。

在这三年里,请保护好你内心那个敏感、脆弱、会哭会笑的小孩。因为等到高考结束,那张试卷会被扔进碎纸机,但这个鲜活的、有血有肉的“你自己”,才是要陪你走完一生的伙伴。

拒绝麻木,哪怕代价是疼痛。因为疼痛,是你还活着的证据。

所谓以人为本,就是解放天性而不是压抑。接纳自己,是一个一生的课题。

\backmatter
\chapter{后记}
写在风暴结束之后

当你读完这本《指北》时,或许你正坐在高一的教室里对未来感到迷茫,或许你正处在高三的暴风眼中心力交瘁。我们写下这些文字,不是为了告诉你所谓什么“高考很简单”,而是想告诉你:我们懂你。

我们懂那种被排名勒住喉咙的窒息感,懂那种看着满天繁星却觉得自己渺小如尘埃的虚无感。这本册子,是我们送给过去的自己,也是送给现在的你的一份礼物。

我们要感谢我们的父母和老师。尽管书中有不少篇幅在教大家如何与他们“博弈”,但我们深知,爱与隔阂往往是共生的。感谢他们的包容,让我们有机会长成现在的样子。


特别感谢本书的编委会成员和所有提供反馈的同学。是大家的真诚,让这本非官方的、充满了“旁门左道”的小书得以诞生。大家在繁重的学业之余,用爱发电,只为了传承一份善意。


正如前言所说,人生是一个混沌系统,一只蝴蝶的振翅可能引起一场风暴。我们希望这本手册,能成为那只蝴蝶。

我们希望你利用书中的“术”,在高考这场游戏中游刃有余;但我们更希望你记住书中的“道”,在分数的洪流中守住那个鲜活的、有血有肉的自己。

高考结束后,这本手册的使命就结束了。请把它留在你的抽屉里,或者传给下一届学弟学妹。然后,请你轻装上阵,去拥抱那个不仅有题海,更有大海的广阔世界。

正如王尔德所说:“把你的生命献给无知、平庸和低俗是病态的。” 去活出你的生命力吧。

祝你好运。
\begin{flushright}
    野草 ~负熵 ~~~~~~~~~~\\survivewhyz@163.com \\2026年1月13日~~~
\end{flushright}


\end{document}