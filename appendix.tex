\documentclass[lang=cn,11pt,bluer,hazy]{elegantbook}

\usepackage{tikz}
\usepackage{pgfplots}
\pgfplotsset{compat=1.18}

\begin{document}
\renewcommand{\today}{\number\year 年 \number\month 月 \number\day 日}
%\maketitle
\begin{titlepage}
    % 1. 页面设置
    \newgeometry{top=4cm, bottom=3cm, left=3cm, right=3cm}
    
    % 2. 顶部标题区 (字体修改重点)
    \begin{center}
        % 主标题:New 序言
        % \heiti: 切换为黑体 (对应图片中的字体)
        % \bfseries: 加粗
        % \fontsize{42pt}{50pt}: 稍微加大字号,增强冲击力
        {\fontsize{42pt}{50pt}\bfseries\lishu\color{black!90} 实验部生存指北:附录 \par}
        
        \vspace{0.8cm}
        
        % 副标题:生存手册
        % 同样使用黑体,去掉装饰线,显得更干净
        {\fontsize{24pt}{30pt}\bfseries\lishu\color{gray!80} —— 做题家自救指南 —— \par}
    \end{center}
    
    \vspace{1.5cm} 
    
    % 3. 核心图形区域 (洛伦兹吸引子)
    \begin{center}
        \begin{tikzpicture}
            % A. 插入 Python 生成的图片 (请确保 lorenz_attractor.png 在文件夹中)
            \node[anchor=center, inner sep=0] (image) at (0,0) {
                \includegraphics[width=12cm]{cover.png}
            };
            
            % C. 底部公式:洛伦兹方程组
            \node[below, opacity=0.7, font=\small, color=black!70] at (0, -4.5) {
                $\begin{cases} 
                \frac{dx}{dt} = \sigma(y-x) \\ 
                \frac{dy}{dt} = x(\rho-z)-y \\ 
                \frac{dz}{dt} = xy-\beta z 
                \end{cases}$
            };
        \end{tikzpicture}
    \end{center}
    
    \vfill
    
    % 4. 底部信息
    \begin{center}
        % 作者名也统一用黑体,风格更统一
        {\large\heiti \authorname 野草 \par}
        \vspace{0.5cm}
        {\heiti\color{gray} \today \par}
        \vspace{0.2cm}
        {\tiny\color{gray!40} \scshape Version 1.3  \par}
    \end{center}
    
    \restoregeometry
\end{titlepage}
% 前言部分
\chapter*{\textit{Love Letters}}

最后的最后,简单谈谈我的高中的一段经历吧。照我的旧例,写成书信的格式。

我不是劝大家谈恋爱,只是想告诉大家,人生中有些东西,是难以用理性去衡量的。不过人间好物不坚固,彩云易散琉璃脆。又有话说叫:“情深不寿”。再美好的恋爱,大概率也是以分手结尾。

与其让恋爱影响你的高三,影响你的学习,不如想想:“……总有一天你会明白,前途比爱情重要,爱情比前途更难得,但是对的人一定会站在你的前途里。”这句话对吗?我不知道,但大概的确没多少人值得你去用前途换一个不确定的未来。祝你好运吧。

对于我的那个“她”,找不到一个合适的身份来叙旧,也就免去那些寒暄了。

\begin{center}
    风起于青萍之末
\end{center}

故事发生在两个普通的高中生之间。

开始的开始,是源自一条qq的生日祝福。我们以前很少说话,但她那天突然发来一句生日快乐,还附了一个吃元宝,一个方了的小表情。

我说谢谢谢谢,从来没想到她会给我发消息,于是后来也就多留意了她一些。

后来事情就慢慢有了进展。她坐在同一大排,我前面两桌。故事是从高二开始的。那时候我还是个路人甲,每天把自己放在题堆里,看着窗边的光打在你脸上。那时候你戴着口罩,眼神直勾勾的,像是在看我,又像是在看穿我身后那面光影斑驳的墙。

等到了你的生日,在高中的教室里徘徊踱步,一直不敢去送。等到班里人越来越多,才鼓起勇气去假装不经意的搭讪上,只是给了你一个本。毕竟我们还没什么关系,太贵重太轻巧也不合适,只好在上面落下轻飘飘几句话,祝你生日快乐。但其实你送我一句生日祝福,我送你好几句,也算公平了罢。

我这种人,平常只敢看着你的背影发呆,却因为那一点点暗中的视线往来,就开始幻想我们能不能有一个未来。

那时候的我,总是那样小心翼翼,甚至为了搞懂什么是“喜欢”,还傻傻地去买了一本《亲密关系》来读 。书上说对视的时候不能先移开目光,可那时的我,光是和你对视一眼,就已经花光了所有的勇气 。

你大概早就猜到了吧?那些为了和你搭话而编造的蹩脚理由,那些放学还要硬凑上去的同路。我就像个拙劣的演员,在聚光灯下演着一出没人看的独角戏,而你是唯一的观众,似笑非笑地看着我:两个人心照不宣地在继续。

那时候,我一天中最快乐的时光,就是骑车在唯一同路的岔路口等候红绿灯的时候四处眺望 。哪怕只能透过车窗玻璃看到一个模糊的影子,甚至有时候什么都看不见,但我心里知道你在那里,那样的一天就会变得充满期待。

那时的日子充满了一种笨拙的期待。我最喜欢的事情,就是把“再见”改成“明天见” 。这样从现在开始,我就有了期待第二天的理由 。在所有的道别里,我最喜欢明天见 。

还记得告白那天吗?我记得。是 5 月 20 号那天,其实也没有特意挑选日子;只是到了时候,不说就要错过了。那天是月考的最后一天,最后一科是数学。等到着急地做完卷子,检查一遍后,就开始在草稿纸上斟酌告白的词句。我还记得第一句话,是“要向你道歉,等了这么久才察觉到你的目光。”最后一句话是“我怕再不表白,就错过了”。演练了好几遍,总感觉能顺畅的背下来,可在路上说的时候,心跳的节拍露了怯,于是背好的词也忘了,出了洋相。过好在你笑了,我知道成功了。

其实我最喜欢的,就是在前后桌坐着的高中生活的每一天。偶尔在课桌下触碰双手交换着体温,或是冬天时作为暖炉给你暖手,轻轻摩挲着指肚,替你吻一吻手上由于握笔太重的茧子,都是珍贵的记忆。

好吧,我坐不了你的后桌:看见你的背影总是走神,于是只能坐前面。午睡结束我总是早早回来坐在你旁边,握住你的手——它只有在这时候才暖暖的,很安心;然后陪你一会,等你起床;或是回来看到你已经在做题了,就担心你会不会睡不够。

你总是一个很优秀的人啊。

\begin{center}
    记忆里的画面,似乎总是在下雨。
\end{center}

你还记得吗?不管是第一次靠近,还是后来的约会,似乎总是在雨天。

可能真的和我的名字对上了,也许我的恋人就是要在雨天来见我 。

第一次靠近你,是在那场突如其来的雨里。你问我要不要一起打伞,我甚至因为不想麻烦别人而下意识拒绝,但还好,最后我还是走进了那把伞下 。那是第一次,我离你那么近,空气是潮湿的,心跳是温热的,连被淋湿的裤脚都成了某种甜蜜的印记 。

我们从没一起出去玩过,约会的地方只是图书馆。在雨幕中,远远望见一个高挑而优雅的单手撑伞的影子映在水中,也映在我的心里,成了一直的印象 。哪怕到了现在,当我鼓起勇气向前看,面前浮现的,依然是在雨中高挑而优雅执伞的你 。

\begin{center}
    拥抱,弥补了右胸腔缺失的心跳
\end{center}

在每个星期天的欣喜里,在图书馆昏暗的走廊中,我们拥抱 。

人们说拥抱是为了弥补人类右胸腔缺失的心跳,的确如此 。那种强烈的满足感填满了每一寸的空虚,烧掉了理性的大脑 。

那段时光里,不仅有雨声,还有糖的甜味。我记得那时候我身体不好要喝苦药,你总会给我准备糖 。每天除了早晚,我最惦记的就是中午的那颗糖和浅浅的拥抱,缓解了嘴里的苦味,也缓解了高三心里的酸涩 。

当然还有些时间回宿舍炫耀一下,每次看到羡慕嫉妒恨,或是夸你的言语,总是替你开心,替你高兴。

还记得我们的高光时刻,在争吵后的月考里,双双拿下前20——不止一次。我只是讨厌学校的照片墙,早不换晚不换——我们两个人好像还没有同时在那个墙上出现过,这是我的一个不大不小的遗憾。

后来我们一直在相互写信,从刚开始的偏正式的小信,到后面逐渐随便的便利贴,留在纸盒里堆成我高中最美好的回忆。我还记得那些开头,展信佳、见字如面、亲爱的xx\dots 还有结尾,纸短情长,吻你万千。

抱歉,人无法共情过去的自己,不过当时的他,是真的很喜欢你。
\begin{center}
    最后一课,我们在天台告别
\end{center}
快乐的时光总是短暂的 。我们也曾经历过争吵、冷漠,经历过那些“沉默占据了全部空白”的日子 。

这些事,我不愿意去提了,也记不清了,只记得对你对我都是一种痛苦和伤害。所以分开,或许是最好的结局。

直到我们在天台分开的那一刻,我忽然想起来,向你告白时的悸动 。虽然有些遗憾,不过是“知不可乎骤得,托遗响于悲风” 。

如今我在吉林,你在浙江。你送我的那枚书签,上面写着“求是创新” 。看着它,我就觉得冥冥之中似乎有什么注定了一般 。

上了大学,分开了一段时间,我还是不习惯没有你的生活,而你还是会出现在我的梦里,甚至比在一起时还要多的,各式各样的梦。过去的恋人,是如今的陌生人
吗?我不知道。不愿意,也不敢去想这些事。看见熟悉的书包,类似的发型,闻到熟悉
的香味,过去的事情就又会浮上心头,原来过去做过的事情也算是在一起吃到的最后
一颗糖,在彼此的口中,留一些酸涩的味道吗?

\begin{center}
    想说的话,都藏在风里
\end{center}

或许是记忆的罅隙被时光扩大,又或是细小的缺点和冲突被无形中抹除,我总是对自己苛责太多,又对你希冀太多。不过我还铭记着那些在床上辗转反侧,高兴的难以入眠的快乐日子。或者执手相看泪眼,无语凝噎的分别的日子。又或者在宿舍床上,静静的流泪,到后来哭的说不出话的日子。酸甜苦辣,喜怒哀乐,百味俱全。
我记得你去浙大研学,要去七天,我就提前写了七篇信,这样你每天就能看一篇,不至于没法消遣。每天最期待的就是等你发下手机,能与我闲聊几句,但是你总是在打牌,没什么时间理我。
我深知我们对彼此并非无可替代,身上的美好品格别人也会具备。但正是花在玫瑰上的时间,让这束玫瑰如此特别。正是你与我度过的美好的日子,让你在我心里如此美丽而优雅,哪怕真实的你来到我的面前,也绝无法与她相提并论。
对了,最近长春下雪了,乌云遮蔽天空,只能偶尔望月。你在杭州和我看到的,是同一轮圆月吗?

谢谢你参与到我的生活,让我的高中生活灰色的日常染上一抹金边 。谢谢你包容我的那些敏感和多疑,谢谢你肯定我的一切,谢谢你曾经喜欢我 。

如果现在的我回到原点,结局还会是一样的吗?如果知道了结局,你还会开始这一切吗? 我想,我的回答是:会 。

大家总告诉我要向前看,但当我鼓起勇气向前看,面前却是在雨中高挑而优雅执
伞的你,是在昏暗灯光下斟酌词句的我,是依偎在一起的身影。我伸手去抓,但那些
身影都化为泡影,在寒风中流逝了。

最可惜的事情莫过于不知道什么时候是最后一次拥抱 。如果当时知道那是最后一次,我一定会多抱一会儿吧 。要是能和你牵着手去看看海就好
了。

但没关系了。 祝你幸福 。

此去经年,应是良辰好景虚设,便纵有千种风情,更与谁人说?
\newpage

\chapter*{后记}
我们一共谈了579天。

真是有点漫长的回忆啊。这些文字是在25年的春天写下的,那时分手也过了一段日子;但莫名的从头回忆了一遍,她就又出现在我的梦里了,像刚开始的时候那样。

说实话,我现在已经不记得喜欢一个人的感觉是什么样子的了,感觉从头去了解一个人是一件很麻烦的事情,我讨厌麻烦。

在刚开始这段关系的时候我从没考虑过分手的事情,看《亲密关系》的时候都跳过了《亲密关系的破裂与修复》的这一章,没想到还是太年轻了。

在开始这段关系之前,我还是把可能遇到的麻烦想的太简单了,或者说我还太幼稚了。回头看看,我无法去埋怨我的父母和老师,他们的确是为了我好;我也不埋怨我曾经的恋人,这不是我的风格,她也没什么错的地方。

那只能埋怨我了;但那时候的他也很可怜,站在雾里也很无助。但他又太单纯了,就连那些简单的对话都会吃醋;也太自我为是,总觉得自己是对的,难以理解对方的感受;又太自卑,没有什么安全感,怕她舍我而去。

但我感觉,这样一种全心全意去喜欢一个人的日子不会再来了;现在算是成熟了一点点,理智的大脑又占据上风了;这样的恋爱脑大概是初恋的专属吧。

我的记忆里她不是那种很漂亮的人,但很有气质,很可爱,很温柔;她学习很用功很认真,字也写得好看;她也很包容我。

从刚开始的信到后面习惯在便利贴上画一些可爱的小表情,或者手工折一个花来相送。这大概是我这辈子第一次收到花。

我们在学校的时候总是相互打气,互相促进的。有时候看见她学的那么用功,也就想多花些时间陪她。看她迷迷糊糊的表情,除了觉得可爱以外也是感到心疼。可能习惯在一起之后,更多的是心疼吧。

我已经好久没见过她了,现在的她大概也没有我心中的那个白月光美好了吧。现在的回忆,与其说是回忆具体的人,不如说是回忆一段时光,回忆那个更有冲劲,更加青春而有生命力的我,与那些一去不返的黄金日子。

拜拜啦,祝你开心和幸福。

\chapter*{兰因絮果,语断难收}
不知道该说什么,分手已经一年多了。

分手是我说的,理应我早把你忘了。但现在不仅没忘,好像还更深了些。往日种种,经常闪回在我的脑中。从你发来生日快乐的喜悦,到出去上网课的天台,图书馆午睡的睡颜,到你的泪眼,我的哭喊,一切都是这么清晰。

执手相看泪眼,竟无语凝噎。

其实我在工作日和忙的时候还好啦,那时候很忙,没什么时间想东想西,也没时间想你,我的惯性很好,做事情还能坚持一段时间来着。

所以我其实有时候不太喜欢放假,一闲下来就陷入了这种无边无际的,长久的空虚和悲伤之中。时不时地,感到自己真的存在吗,还活着吗,这是在做梦吗?我好难过啊。

恋爱只是人生中的一小部分,我承认是这样的,但对我不是。我没办法忍受这种长久的空虚,如果我不曾感到充实;又没办法随便去找一个对象——尽管我大概率也找不到——来随意的,假装填满了自己。

轩问我怎么办,难道这辈子就不谈恋爱了吗?我的回答是大概率不是这样,但我现在真的很难捱,但我打算捱一阵子再说。我想约你出来见一面,我受不了了,但我担心你根本不理我,也就没有发送。我好难过啊,但也只能捱着,没什么人能倾诉。

真的是兰因絮果,语断难收。

吵架之后,我还记得抱住你的感受,你抓着我的衣服,头靠在我的身上,眼泪很烫。我嘴一直很笨,只在没用的场合会说话,到了安慰人的时候就哑口无言了。

我喜欢捧着你的脸亲上去,捏捏你软软的脸。以前总怕没有话题可以说,哪怕尬聊我也要去找些话题来聊;现在想想,不聊天呆在一起也很快乐。一去不复返了,坐在图书馆的台阶上,听你和我说想吃什么甜品,在手机上看美团的日子。

还有高二唯二的音乐课,讲了巴赫和他的复调;下了课你给我递纸条,问我有没有认真听?给我简单讲了讲什么是复调;你体育成绩出了问题,去西院找老师,我也赶紧赶过去,到了的时候,你满脸都是泪,眼眶红的吓人,我好心疼;但也搭不上话,只能在回去的路上安慰你。

我讨厌你,和讨厌我一样,我讨厌你。你现在不回我了,就像我当时拒绝了你一样,没什么可以置喙的。

距离上次联系你,感觉上又过去了半年,一个学期,不同的是你连理我都不理了。

除了感情以外,我自认做的都不是很差,只有在这件事情上什么都没有做对过一样,成了情感中的阶下囚。

姥姥那天还问我,是不是还谈着呢?故作轻松的说没有,其实还是很难过的。我想哭。

我想见见你,哪怕这不是正确的事。

\end{document}